% Based on tufte-handout.tex - DESC
% Iago Mosqueira - JRC. 2013
% https://gist.github.com/reinholdsson/7426608/566d8104363cd2fef2d6dd50e90aceab377b4a45

% Example input file: https://t.co/Jl73l6p8rA
% Example output file: https://t.co/udpLlvml8X

\documentclass{tufte-handout}

\providecommand{\note}{\textsf}
\renewcommand{\large}{\normalsize}
\renewcommand{\huge}{\normalsize}
\renewcommand{\Huge}{\normalsize}

% optionally, change all figures to margin figures (Caleb McDaniel)

% optionally, make links footnotes instead of hotlinks
% 
% ams
\usepackage{amssymb,amsmath}

% add line numbers (Caleb McDaniel)

% use symbols instead of numbers for footnotes (Caleb McDaniel)
% http://tex.stackexchange.com/questions/826/symbols-instead-of-numbers-as-footnote-markers
% \renewcommand*{\thefootnote}{\fnsymbol{footnote}}

% Set up the images/graphics package
\usepackage{graphicx}
\setkeys{Gin}{width=\linewidth,totalheight=\textheight,keepaspectratio}
\graphicspath{{graphics/}}

\let\Oldincludegraphics\includegraphics
\renewcommand{\includegraphics}[1]{\Oldincludegraphics[trim={0 3in 0 0},clip,width=\textwidth]{#1}}

% natbib
\usepackage{natbib}
\bibliographystyle{plainnat}

% biblatex

% booktabs
\usepackage{longtable,booktabs}

% url
\usepackage{url}

% hyperref
\usepackage{hyperref}

% units.
\usepackage{units}

% fancyvrb
\usepackage{fancyvrb}
\fvset{fontsize=\normalsize}
\DefineShortVerb[commandchars=\\\{\}]{\|}
\DefineVerbatimEnvironment{Highlighting}{Verbatim}{commandchars=\\\{\}}


% multiplecol
\usepackage{multicol}

% lipsum
\usepackage{lipsum}

% These commands are used to pretty-print LaTeX commands
\newcommand{\doccmd}[1]{\texttt{\textbackslash#1}}% command name -- adds backslash automatically
\newcommand{\docopt}[1]{\ensuremath{\langle}\textrm{\textit{#1}}\ensuremath{\rangle}}% optional command argument
\newcommand{\docarg}[1]{\textrm{\textit{#1}}}% (required) command argument
\newenvironment{docspec}{\begin{quote}\noindent}{\end{quote}}% command specification environment
\newcommand{\docenv}[1]{\textsf{#1}}% environment name
\newcommand{\docpkg}[1]{\texttt{#1}}% package name
\newcommand{\doccls}[1]{\texttt{#1}}% document class name
\newcommand{\docclsopt}[1]{\texttt{#1}}% document class option name

% Shaded
\newenvironment{Shaded}{}{}
\newcommand{\KeywordTok}[1]{\textcolor[rgb]{0.00,0.44,0.13}{\textbf{{#1}}}}
\newcommand{\DataTypeTok}[1]{\textcolor[rgb]{0.56,0.13,0.00}{{#1}}}
\newcommand{\DecValTok}[1]{\textcolor[rgb]{0.25,0.63,0.44}{{#1}}}
\newcommand{\BaseNTok}[1]{\textcolor[rgb]{0.25,0.63,0.44}{{#1}}}
\newcommand{\FloatTok}[1]{\textcolor[rgb]{0.25,0.63,0.44}{{#1}}}
\newcommand{\CharTok}[1]{\textcolor[rgb]{0.25,0.44,0.63}{{#1}}}
\newcommand{\StringTok}[1]{\textcolor[rgb]{0.25,0.44,0.63}{{#1}}}
\newcommand{\CommentTok}[1]{\textcolor[rgb]{0.38,0.63,0.69}{\textit{{#1}}}}
\newcommand{\OtherTok}[1]{\textcolor[rgb]{0.00,0.44,0.13}{{#1}}}
\newcommand{\AlertTok}[1]{\textcolor[rgb]{1.00,0.00,0.00}{\textbf{{#1}}}}
\newcommand{\FunctionTok}[1]{\textcolor[rgb]{0.02,0.16,0.49}{{#1}}}
\newcommand{\RegionMarkerTok}[1]{{#1}}
\newcommand{\ErrorTok}[1]{\textcolor[rgb]{1.00,0.00,0.00}{\textbf{{#1}}}}
\newcommand{\NormalTok}[1]{{#1}}

% Support pandoc's -H/--include-in-header option

\title{Text Technologies for Data Science Assignment 4}
\author{s1140740}

\begin{document}
\maketitle

\bigskip


\section{PageRank Algorithm}\label{pagerank-algorithm}

People are represented as nodes with a directed edge from a person
sending an email to a person receiving the email. The weight ($w$) of
this edge represents the number of emails the person sent. PageRanks
($PR$) are initialised to $1/N$ where $N$ is the total number of people.
10 iterations of the algorithm are then executed. Each iteration $S$,
the sum of PageRanks of sink nodes, is computed and people's PageRanks
are updated using the PageRank formula.\footnote{\[PR(x) = \frac{1-\lambda+\lambda S}{N} + \lambda \sum_{y \rightarrow x} \frac{w \cdot PR(y)}{out(y)} \]}

After running the algorithm for 10 iterations, \texttt{klay@enron.com}
turned out to be the email address with the highest PageRank of
0.008027. Moreover, the email address in 6th place is
\texttt{kenneth.lay@enron.com} which is probably the email address of
the same person, CEO of Enron, Kenneth Lay. We can thus conclude that
the PageRank algorithm did quite well in figuring out who the
influential people are. The top 5 PageRanks can be found in table 1.

\begin{longtable}[c]{@{}ll@{}}
\toprule\addlinespace
\begin{minipage}[b]{0.37\columnwidth}\raggedright
Email
\end{minipage} & \begin{minipage}[b]{0.14\columnwidth}\raggedright
PageRank
\end{minipage}
\\\addlinespace
\midrule\endhead
\begin{minipage}[t]{0.37\columnwidth}\raggedright
klay@enron.com
\end{minipage} & \begin{minipage}[t]{0.14\columnwidth}\raggedright
0.008027
\end{minipage}
\\\addlinespace
\begin{minipage}[t]{0.37\columnwidth}\raggedright
jeff.skilling@enron.com
\end{minipage} & \begin{minipage}[t]{0.14\columnwidth}\raggedright
0.003019
\end{minipage}
\\\addlinespace
\begin{minipage}[t]{0.37\columnwidth}\raggedright
sara.shackleton@enron.com
\end{minipage} & \begin{minipage}[t]{0.14\columnwidth}\raggedright
0.002961
\end{minipage}
\\\addlinespace
\begin{minipage}[t]{0.37\columnwidth}\raggedright
tana.jones@enron.com
\end{minipage} & \begin{minipage}[t]{0.14\columnwidth}\raggedright
0.002855
\end{minipage}
\\\addlinespace
\begin{minipage}[t]{0.37\columnwidth}\raggedright
mark.taylor@enron.com
\end{minipage} & \begin{minipage}[t]{0.14\columnwidth}\raggedright
0.002753
\end{minipage}
\\\addlinespace
\bottomrule
\addlinespace
\caption{Top 5 PageRanks}
\end{longtable}

\section{HITS Algorithm}\label{hits-algorithm}

Same representation as for PageRank is used. However, hub and authority
values are initialised to $1/\sqrt{N}$. Each iteration the hub value of
a node is updated using authority values of nodes it is pointing
to\footnote{\[H(x) = \sum_{y \leftarrow x} A(y)\]} and the authority
value is updated using hub values pointing to it.\footnote{\[A(x) = \sum_{y \rightarrow x} H(y)\]}
Both hub and authority values are then normalised.\footnote{\[\sum_{x} H(x)^2 = 1 = \sum_{x} A(x)^2\]}

After running the algorithm for 20 iterations,
\texttt{pete.davis@enron.com} turned out to be the email address with
the highest hub score of 0.999281. This is consistent with his role
description in \texttt{roles.txt} - broadcast proxy for auto-generated
emails. Other than that there were no abnormalities. Email addresses
with top 5 hub scores and top 5 authority scores are in tables 2 and 3,
respectively.

\begin{longtable}[c]{@{}ll@{}}
\toprule\addlinespace
\begin{minipage}[b]{0.24\columnwidth}\raggedright
Email
\end{minipage} & \begin{minipage}[b]{0.16\columnwidth}\raggedright
Hub score
\end{minipage}
\\\addlinespace
\midrule\endhead
\begin{minipage}[t]{0.24\columnwidth}\raggedright
pete.davis
\end{minipage} & \begin{minipage}[t]{0.16\columnwidth}\raggedright
0.999281
\end{minipage}
\\\addlinespace
\begin{minipage}[t]{0.24\columnwidth}\raggedright
bill.williams
\end{minipage} & \begin{minipage}[t]{0.16\columnwidth}\raggedright
0.032970
\end{minipage}
\\\addlinespace
\begin{minipage}[t]{0.24\columnwidth}\raggedright
rhonda.denton
\end{minipage} & \begin{minipage}[t]{0.16\columnwidth}\raggedright
0.010408
\end{minipage}
\\\addlinespace
\begin{minipage}[t]{0.24\columnwidth}\raggedright
l..denton
\end{minipage} & \begin{minipage}[t]{0.16\columnwidth}\raggedright
0.006774
\end{minipage}
\\\addlinespace
\begin{minipage}[t]{0.24\columnwidth}\raggedright
grace.rodriguez
\end{minipage} & \begin{minipage}[t]{0.16\columnwidth}\raggedright
0.005825
\end{minipage}
\\\addlinespace
\bottomrule
\addlinespace
\caption{Top 5 Hub scores}
\end{longtable}

\begin{longtable}[c]{@{}ll@{}}
\toprule\addlinespace
\begin{minipage}[b]{0.21\columnwidth}\raggedright
Email
\end{minipage} & \begin{minipage}[b]{0.17\columnwidth}\raggedright
Auth score
\end{minipage}
\\\addlinespace
\midrule\endhead
\begin{minipage}[t]{0.21\columnwidth}\raggedright
ryan.slinger
\end{minipage} & \begin{minipage}[t]{0.17\columnwidth}\raggedright
0.384187
\end{minipage}
\\\addlinespace
\begin{minipage}[t]{0.21\columnwidth}\raggedright
albert.meyers
\end{minipage} & \begin{minipage}[t]{0.17\columnwidth}\raggedright
0.384177
\end{minipage}
\\\addlinespace
\begin{minipage}[t]{0.21\columnwidth}\raggedright
mark.guzman
\end{minipage} & \begin{minipage}[t]{0.17\columnwidth}\raggedright
0.383849
\end{minipage}
\\\addlinespace
\begin{minipage}[t]{0.21\columnwidth}\raggedright
geir.solberg
\end{minipage} & \begin{minipage}[t]{0.17\columnwidth}\raggedright
0.383764
\end{minipage}
\\\addlinespace
\begin{minipage}[t]{0.21\columnwidth}\raggedright
craig.dean
\end{minipage} & \begin{minipage}[t]{0.17\columnwidth}\raggedright
0.355581
\end{minipage}
\\\addlinespace
\bottomrule
\addlinespace
\caption{Top 5 Auth scores}
\end{longtable}

\section{Visualizing key connections}\label{visualizing-key-connections}

I visualised connections between 10 people with the highest PageRank. It
seemed that the algorithm ranked people with great influence highly and
thus visualising their connections could provide some additional
information about the scandal. I used the networkx library\footnote{NetworkX,
  https://networkx.github.io} to create a graph of these connections
which is then outputted into a file \texttt{graph.dot} and visualised
with graphviz.\footnote{graphviz, http://www.graphviz.org}

I used information from \texttt{roles.txt} to assign names and roles to
email accounts of Enron employees. Furthermore, the number of emails
exchanged between two people is displayed at the tail of an edge in red
colour.

My algorithm does not require any additional manual tuning which means
that it is very automatic but that it is also quite simple.

\end{document}
